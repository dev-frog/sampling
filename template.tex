\documentclass{report}



\input{preamble}
\input{macros}
\input{letterfonts}


\title{REVIEW ON SAMPLING METHODOLOGY OF A SURVEY\\Child and Monther Nutrition Survey 2012}
\author{MD.Minjurul Haque \\ID: 20231134}
\date{}

\begin{document}

\maketitle
% \newpage%
% \pdfbookmark[section]{\contentsname}{toc}
% \tableofcontents
\pagebreak



\chapter{\huge{INTRODUCTION}}
\section{Child and nutrition status}


Millions of children and women in Bangladesh endure various forms of malnutrition, evident in low birth weight, wasting, stunting, underweight, Vitamin A deficiencies, iodine deficiency disorders, and anemia. This cycle perpetuates as malnourished mothers give birth to infants who struggle to thrive, particularly affecting future generations.Despite progress in achieving Millennium Development Goals (MDG) like eradicating extreme poverty and hunger, addressing child mortality, and improving maternal health, more efforts are needed. MDG1, MDG4, and MDG5 are intricately linked to nutrition. Enhancing nutrition, especially for under-5 children and mothers, is crucial for achieving MDGs. 
Malnourished mothers face higher risks during pregnancy, while malnourished children exhibit lower resistance to infections, often succumbing to common ailments.Nutrition plays a pivotal role in development, with growth faltering in children typically beginning around four to six months when complementary foods are introduced alongside breastfeeding. This deviation from normal growth results from a combination of poor nutrition, intra-uterine growth restriction, and exacerbated morbidity, particularly from diarrheal diseases. Inadequate complementary foods and caregiving practices contribute to early childhood growth retardation, with lasting implications on the onset of growth spurts and subsequent development. 
The process of growth faltering unfolds as a series of events throughout childhood, marked by recurring illnesses, insufficient appetite, deficient nutrition, and substandard care. Many children facing these challenges experience premature mortality, while survivors bear long-term consequences such as stunted growth and compromised mental capacity. Addressing these issues is paramount for breaking the cycle of malnutrition and promoting sustainable development in Bangladesh. The population policy of the Government of Bangladesh aims to achieve several key goals related to child and maternal nutrition. These include reducing infant and under-five mortality rates, improving maternal health with a specific focus on decreasing maternal mortality, implementing childhood development programs, and addressing the issue of child and maternal malnutrition.



\section{Background}


The Bangladesh Bureau of Statistics has conducted the Child Nutrition Survey (CNS) for several years, with the sixth survey in 2005 being the first to include both mothers and children. The seventh survey, Child and Mother Nutrition Survey (CMNS) in 2012, aimed to gather up-to-date nutrition data for both demographic groups. Despite no specific surveys after 2005, the Bureau recognized the importance of CMNS and conducted it during the 2011-12 financial years to generate recent data. Field-level data collection occurred from 7th to 10th March 2012, following the preparation of a list of under-five children from each of 30 households between 29th February and 6th March 2012.

\pagebreak

\section{Objectives of the CMNS 2012}
The objectives of the CMNS 2012 included the following targets:
\begin{enumerate}
\item To measure the nutritional status (underweight, stunting, wasting, and overweight) of
children aged 00-59 months (under 5 years of age).
\item To measure the nutritional status of mothers of children aged 00-59 months.
\item To assess knowledge, attitude and practices related to child and maternal care that affect
their nutritional status including infant and young child feeding and care practices during
illness.
\item To asses household food security.
\item To collect data on demographic, socio-economic and cultural variables (Household
Characteristics, education etc.)
\item To collect data on measles vaccination, feeding Vitamin ‘A’ capsule, night blindness or
clinical Vitamin A deficiency (18-59 months of age) of children 

\end{enumerate}

\chapter{\huge{METHODOLOGY}}
\section{Survey area and population}

The Child and Mother Nutrition Survey (CMNS) 2012 was conducted on a nationally representative sample encompassing both rural and urban areas, focusing on children aged 0-59 months along with their mothers. 
The survey took place over four days, from March 7th to March 10th, 2012. Prior to the field-level data collection, an extensive preparatory phase spanned seven days, from February 29th to March 6th, 2012.
During this preparatory phase, a meticulous list of under-five children was compiled from each of the 30 households. This comprehensive process unfolded over seven days, ensuring a thorough representation of the target demographic. 
Subsequently, information regarding both mothers and children was systematically collected from a total of 350 Primary Sampling Units (PSUs), spanning 63 districts and 7 divisions across Bangladesh. 
The inclusion of a nationally representative sample, rigorous preparation, and extensive coverage of PSUs, districts, and divisions underscored the comprehensiveness and reliability of the CMNS 2012 data. 
This approach aimed to provide a nuanced understanding of the nutritional status of children and mothers across diverse geographical and demographic contexts within the country.



\section{Sampling design and sample size }


The Child and Mother Nutrition Survey (CMNS) 2012 focused on a subsample of households surveyed during the Health and Morbidity Status Survey (HMSS) 2012. To ensure representative data at both the national and divisional levels, the sample size for CMNS 2012 was calculated using a specific formula. The application of this formula aimed to optimize the precision and reliability of the survey outcomes, aligning with the overarching goal of obtaining comprehensive insights into the nutritional status of children and mothers across various geographical and demographic strata.

Divisional level for the CMNS 2012 was calculated using the following formula:

\[ n = z^2 \frac{P(1-P)}{d^2} \times D_{\text{eff}} \]



\textbf{Where:}


\begin{itemize}
\item $n$ = Sample size
\item $z$ = Standard normal deviate at 95\% confidence level (1.96)
\item $P$ = Estimated prevalence of underweight children (0.40)
\item $d$ = Precision (0.05)
\item $D_{\text{eff}}$ = Design effect (1.5)
\end{itemize}

\pagebreak

\textbf{Stunting Percentage (P):} The assumption is that the prevalence of stunting (a measure of chronic malnutrition, usually defined as low height-for-age) is estimated to be 50\% This is considered a conservative estimate, indicating that the survey is preparing for a scenario where a significant portion of the surveyed population may be affected by stunting.

\textbf{Precision (d):} The precision, often denoted as 'd', refers to the margin of error allowed in the survey results. In this case, a precision of 5\% is chosen, meaning that the calculated estimates for stunting prevalence should be within plus or minus 5 percentage points of the true value. A lower precision indicates a more accurate and detailed estimate.

\textbf{Confidence Level (95\%):} The confidence level is the probability that the true parameter (here, stunting prevalence) falls within the calculated interval. A 95\% confidence level is commonly used, meaning that if the survey were conducted many times, the calculated interval would contain the true stunting prevalence in 95\% of those surveys.

\textbf{Design Effect:} The design effect is a factor that adjusts the sample size formula to account for the complexity of the survey design. In this case, a design effect of 1.7 is used. A design effect greater than 1 indicates that the survey design is more complex than a simple random sample, and adjustments are made to ensure the calculated sample size is sufficient to achieve the desired precision and confidence level.


\section{Sampling frame}
The Child and Mother Nutrition Survey (CMNS) 2012 aimed to gather comprehensive data on child and maternal nutrition in Bangladesh. The survey covered a significant sample, including 4112 children aged 0-59 months and 3521 mothers residing in 3484 households across both rural and urban areas. To ensure national representativeness, the survey focused on a diverse sample of children and mothers from both settings.

In adherence to the survey design formula: 
\[ n = z^2 \frac{P(1-P)}{d^2} \times D_{\text{eff}} \]

the Health and Morbidity Status Survey (HMSS) 2012, which preceded CMNS 2012, surveyed 30 households in each Primary Sampling Unit (PSU). From the extensive coverage of 1000 PSUs in the HMSS, a sub-sample of 350 PSUs was selected for CMNS 2012. In each of these 350 PSUs, the survey targeted 30 sample households to identify children aged 0-59 months and their respective mothers.


A meticulous household listing operation was conducted in all 350 PSUs to establish a robust sampling frame for the second stage of household selection. In this second stage, 30 households were systematically chosen from each PSU based on the household listing. This methodical approach ensured the selection of 10500 households overall, providing a representative and statistically sound foundation for estimating key demographic and nutrition variables. The survey design aimed to generate reliable insights for the entire country, each of the seven divisions, and urban and rural areas separately, aligning with the broader goal of comprehensive and accurate data collection.

\begin{table}[ht]
\centering
\caption{Distribution of PSU for CMNS 2012 and HMSS 2012 by division and area of residence, Bangladesh, 2012}

\label{tab:psu_distribution}
\begin{tabular}{lccccccc}
\toprule
\multirow{2}{*}{Division} & \multicolumn{3}{c}{CMNS 2012} & \multicolumn{3}{c}{HMSS 2012}\\
\cmidrule(lr){2-4} \cmidrule(lr){5-7}
& Rural & Urban & Total & Rural & Urban & Total & \\
\midrule
Barisal & 33 & 17 & 50 & 55 & 25 & 80 \\
Chittagong & 33 & 17 & 50 & 116 & 63 & 179 \\
Dhaka & 32 & 18 & 50 & 172 & 117 & 289 \\
Khulna & 31 & 19 & 50 & 89 & 57 & 146 \\
Rajshahi & 35 & 15 & 50 & 88 & 46 & 134 \\
Rangpur & 33 & 17 & 50 & 82 & 35 & 117 \\
Sylhet & 36 & 14 & 50 & 38 & 17 & 55 \\
\midrule
Total & 233 & 117 & 350 & 640 & 360 & 1000 \\
\bottomrule
\end{tabular}
\end{table}

\pagebreak
\begin{table}[ht]
\centering
\caption{Division and Rural-Urban wise sample allocation of CMNS 2012, Bangladesh}
\label{tab:sample_allocation}
\begin{tabular}{lccccccc}
\toprule
\multirow{2}{*}{Division} & \multicolumn{3}{c}{Number of Sample PSUs} & \multicolumn{3}{c}{Number of Sample SSUs (HH)}  \\
\cmidrule(lr){2-4} \cmidrule(lr){5-7}
& Urban & Rural & Total & Urban & Rural & Total & \\
\midrule
Barisal & 17 & 33 & 50 & 510 & 990 & 1500 \\
Chittagong & 17 & 33 & 50 & 510 & 990 & 1500 \\
Dhaka & 18 & 32 & 50 & 540 & 960 & 1500 \\
Khulna & 19 & 31 & 50 & 570 & 930 & 1500 \\
Rajshahi & 15 & 35 & 50 & 450 & 1050 & 1500 \\
Rangpur & 17 & 33 & 50 & 510 & 990 & 1500 \\
Sylhet & 14 & 36 & 50 & 420 & 1080 & 1500 \\
\midrule
Total & 117 & 233 & 350 & 3510 & 6990 & 10500 \\
\bottomrule
\end{tabular}
\end{table}


\section{Data collection methods}
The data collection methods for the Child and Mother Nutrition Survey (CMNS) 2012 involved the use of a comprehensive questionnaire designed to gather information on various aspects. This included household socio-economic and sociodemographic status, access to health services, the health environment, household food security, caring practices, and anthropometry, which involved measurements such as length/height, weight, and Mid Upper Arm Circumference (MUAC) for children and their mothers.


To ensure precision in measurements, the UNISCALE (Seca, Hamburg, Germany) was employed to measure the weight of both children and mothers, providing accurate readings to the nearest 100 grams. Additionally, a locally crafted wooden height scale was utilized for measuring the length of children under 2 years and the height of mothers and children aged at least 2 years, with measurements recorded to the nearest 1 millimeter. The Mid Upper Arm Circumference (MUAC) of children and women, both pregnant and non-pregnant, was measured with precision to the nearest 0.2 millimeters using a numerical insertion tape manufactured in China. This meticulous approach aimed to capture reliable and detailed anthropometric data, crucial for assessing the nutritional status of the surveyed population.

\section{Selection of sample households}

In the context of Bangladesh, the household data collection for the Child and Mother Nutrition Survey (CMNS) 2012 was intricately connected with the information gathered during the Health and Demographic Survey (HDS) 2012, a significant initiative undertaken by the Bangladesh Bureau of Statistics (BBS). Chapter 3 of the CMNS 2012 report extensively utilized data extracted from HDS 2012, given that CMNS 2012 was conducted in the same geographical areas covered by the earlier survey.

This methodological choice was made in accordance with the decision of the Technical Committee established specifically for surveys conducted under the Demography and Health Wing. By harmonizing the data collection efforts and drawing upon the existing dataset from HDS 2012, CMNS 2012 ensured a streamlined and consistent approach to capturing personal and socio-economic characteristics of household members in the surveyed regions. This decision not only maximized efficiency but also facilitated a more comprehensive understanding of the demographic and health landscape in Bangladesh. It underscores the collaborative and strategic efforts taken by the relevant authorities to optimize the use of existing resources and enhance the overall quality of survey outcomes, ensuring that the collected data is both relevant and contextualized within the broader health and demographic context of Bangladesh.


\section{Data collection process}

The data collection process for the Child and Mother Nutrition Survey (CMNS) 2012 in Bangladesh was a meticulously planned and executed effort aimed at gathering comprehensive information on household socio-economic and socio-demographic status, health-related factors, food security, caring practices, and anthropometric measurements of children and their mothers.

\subsection{Survey Design and Instrumentation:}
The survey utilized a structured questionnaire (Annex-2) designed to capture a wide range of relevant data. The questionnaire covered key aspects such as household demographics, socio-economic status, health service access, environmental factors, food security, and anthropometric measurements, including length/height, weight, and Mid Upper Arm Circumference (MUAC) for both children and mothers.

\subsection{Anthropometric Measurements:}
To ensure accuracy in measuring the weight of children and mothers, the UNISCALE (Seca, Hamburg, Germany) was employed. This sophisticated equipment provided precise weight measurements to the nearest 100 grams. For the measurement of length in children under 2 years and height in mothers and children aged at least 2 years, a locally crafted wooden height scale was used. The Mid Upper Arm Circumference (MUAC) of children and women, whether pregnant or non-pregnant, was measured using a numerical insertion tape manufactured in China, with measurements recorded to the nearest 0.2 millimeters.

\subsection{Survey Team Composition:}
The data collection teams were composed of three members, ensuring both gender diversity and a combination of field and head office representation. Each team comprised one male member from the Upazila Statistical office, one male member from the head office, and a female member known as the local register, responsible for collecting anthropometric information.

\subsection{Supervision and Coordination:}
A supervising officer from both the head office and field offices played a critical role in overseeing and coordinating the survey activities. These officers were responsible for specific districts, ensuring consistency, quality control, and adherence to survey protocols. Their supervision contributed to the reliability and integrity of the collected data.

\subsection{Integration with Health and Demographic Survey (HDS) 2012:}
The decision to integrate CMNS 2012 data collection with the existing dataset from the Health and Demographic Survey (HDS) 2012 provided several advantages. Conducting both surveys in the same areas ensured geographical continuity and facilitated a comprehensive understanding of demographic and health-related trends. This integration was driven by the Technical Committee established for surveys under the Demography and Health Wing, demonstrating a strategic approach to optimize resources and enhance the overall quality and relevance of the data.

\subsection{Conclusion:}
The data collection process of CMNS 2012 in Bangladesh exemplified a collaborative and well-coordinated effort, incorporating advanced measurement techniques, gender-diverse teams, and integration with previous survey data. This approach not only ensured the collection of robust and detailed information but also contributed to the broader understanding of the nutritional landscape and health dynamics in Bangladesh. The careful planning, utilization of advanced tools, and integration strategies underscore the commitment to producing reliable data for informed decision-making in the realm of child and maternal nutrition.



% \chapter{\huge{CONCLUSION}}
\setcounter{secnumdepth}{-1}
\chapter{\huge{CONCLUSION}}


The findings of the survey paint a comprehensive picture of household characteristics and maternal and child caring practices in the surveyed population. The mean household size, higher than national averages, reflects the intentional inclusion of households with young children. Male predominance in household leadership, coupled with significant levels of illiteracy, underscores the need for targeted interventions in education and gender equity. 
Noteworthy disparities persist in access to sanitation and clean water, emphasizing the importance of addressing basic infrastructure needs. The proximity of health facilities, with an average distance of 2.5 km, presents both challenges and opportunities for improving healthcare accessibility. 
Maternal and child caring practices reveal both commendable aspects and areas for enhancement. While knowledge of exclusive breastfeeding and certain immunization practices is evident, gaps exist in antenatal care and nutritional supplementation during pregnancy. Addressing these gaps is crucial for the holistic well-being of mothers and children.


In conclusion, the survey provides valuable insights into the dynamics of households and maternal and child health practices. The data serve as a foundation for informed policymaking and targeted interventions aimed at improving overall health outcomes in the surveyed population. Continued efforts in education, healthcare accessibility, and community-based interventions are essential for achieving sustainable improvements in maternal and child health.



\end{document}


